
\documentclass[6pt,a4paper]{article}
\usepackage[a4paper,margin=0.54cm]{geometry}
\usepackage{longtable}
\usepackage{rotating}
\usepackage{pdflscape}
\usepackage{ctable}
\title{Statistical Analysis in the second study for signed-up students }
\date{}
\begin{document}

\maketitle


%latex.default(result_df, caption = paste("Two-way ANOVA and Scheirer-Ray-Hare",     in_title), size = "small", longtable = T, ctable = F, landscape = F,     rowlabel = "", where = "!htbp", file = filename, append = T)%
\setlongtables{\small
\begin{longtable}{lrrrrlrrrrr}\caption{Two-way ANOVA and Scheirer-Ray-Hare in the second study for signed-up students} \tabularnewline
\hline\hline
\multicolumn{1}{l}{}&\multicolumn{1}{c}{Sum Sq}&\multicolumn{1}{c}{Df}&\multicolumn{1}{c}{F value}&\multicolumn{1}{c}{Pr(\textgreater F)}&\multicolumn{1}{c}{Sig}&\multicolumn{1}{c}{Df}&\multicolumn{1}{c}{Sum Sq}&\multicolumn{1}{c}{H}&\multicolumn{1}{c}{p.value}&\multicolumn{1}{c}{Sig}\tabularnewline
\hline
\endfirsthead\caption[]{\em (continued)} \tabularnewline
\hline
\multicolumn{1}{l}{}&\multicolumn{1}{c}{Sum Sq}&\multicolumn{1}{c}{Df}&\multicolumn{1}{c}{F value}&\multicolumn{1}{c}{Pr(\textgreater F)}&\multicolumn{1}{c}{Sig}&\multicolumn{1}{c}{Df}&\multicolumn{1}{c}{Sum Sq}&\multicolumn{1}{c}{H}&\multicolumn{1}{c}{p.value}&\multicolumn{1}{c}{Sig}\tabularnewline
\hline
\endhead
\hline
\endfoot
\label{result}
difScore.(Intercept)&$ 86.838$&$ 1$&$24.758$&$0.000$&&$$&$$&$$&$$&$$\tabularnewline
difScore.Type&$  1.995$&$ 1$&$ 0.569$&$0.455$&&$ 1$&$  13.612$&$0.069$&$0.792$&$$\tabularnewline
difScore.CLRole&$ 14.418$&$ 1$&$ 4.111$&$0.049$&*&$ 1$&$ 703.592$&$3.590$&$0.058$&$$\tabularnewline
difScore.Type:CLRole&$  0.976$&$ 1$&$ 0.278$&$0.601$&&$ 1$&$  48.644$&$0.248$&$0.618$&$$\tabularnewline
difScore.Residuals&$154.327$&$44$&$$&$$&&$44$&$8446.152$&$$&$$&$$\tabularnewline
\hline
\end{longtable}}
\begin{flushright}{\scriptsize{Signif. codes: 0 ``**'' 0.01 ``*'' 0.05}}\end{flushright}


\begin{landscape}

%latex.default(result_df, caption = paste("Summary of Pair wilcoxon",     in_title), size = "small", longtable = T, ctable = F, landscape = F,     rowlabel = "", where = "!htbp", file = filename, append = T)%
\setlongtables{\small
\begin{longtable}{r}\caption{Summary of Pair wilcoxon in the second study for signed-up students} \tabularnewline
\hline\hline
\multicolumn{1}{c}{}\tabularnewline
\hline
\endfirsthead\caption[]{\em (continued)} \tabularnewline
\hline
\multicolumn{1}{c}{}\tabularnewline
\hline
\endhead
\hline
\endfoot
\label{result}
$$\tabularnewline
\hline
\end{longtable}}

\end{landscape}

\section{Assumptions for Parametric Tests}
%latex.default(round_df(do.call(rbind, lapply(all_parametric_results,     function(p_results) {        return(do.call(rbind, lapply(p_results, function(p_result) {            return(data.frame(normality.fail = p_result$normality.fail,                 W = p_result$shapiro$statistic, p.value = p_result$shapiro$p.value))        })))    })), 3), caption = paste("Univariate normality test", in_title),     size = "scriptsize", longtable = T, ctable = F, landscape = F,     rowlabel = "", where = "!htbp", file = filename, append = T)%
\setlongtables{\scriptsize
\begin{longtable}{llrr}\caption{Univariate normality test in the second study for signed-up students} \tabularnewline
\hline\hline
\multicolumn{1}{l}{}&\multicolumn{1}{c}{normality.fail}&\multicolumn{1}{c}{W}&\multicolumn{1}{c}{p.value}\tabularnewline
\hline
\endfirsthead\caption[]{\em (continued)} \tabularnewline
\hline
\multicolumn{1}{l}{}&\multicolumn{1}{c}{normality.fail}&\multicolumn{1}{c}{W}&\multicolumn{1}{c}{p.value}\tabularnewline
\hline
\endhead
\hline
\endfoot
\label{round}
difScore&FALSE&$0.99$&$0.95$\tabularnewline
\hline
\end{longtable}}

%latex.default(test_min_size_summary, caption = paste("Notes to be taken into account about sample size",     in_title), size = "scriptsize", longtable = T, ctable = F,     landscape = F, rowlabel = "", where = "!htbp", file = filename,     append = T)%
\setlongtables{\scriptsize
\begin{longtable}{lll}\caption{Notes to be taken into account about sample size in the second study for signed-up students} \tabularnewline
\hline\hline
\multicolumn{1}{l}{}&\multicolumn{1}{c}{code}&\multicolumn{1}{c}{description}\tabularnewline
\hline
\endfirsthead\caption[]{\em (continued)} \tabularnewline
\hline
\multicolumn{1}{l}{}&\multicolumn{1}{c}{code}&\multicolumn{1}{c}{description}\tabularnewline
\hline
\endhead
\hline
\endfoot
\label{test}
difScore.Type.1&WARN: sample.size&current size is 14 and recommended size is 15 for the group: 'ont-gamified:Apprentice'.\tabularnewline
difScore.Type.2&WARN: sample.size&current size is 8 and recommended size is 15 for the group: 'non-gamified:Master'.\tabularnewline
difScore.Type.3&WARN: sample.size&current size is 5 and recommended size is 15 for the group: 'ont-gamified:Master'.\tabularnewline
\hline
\end{longtable}}

Recent studies carried out through simulations have indicated that ANOVA is reliable even when the data are non-normally distributed and the sample size is greater thatn 15 observations for each group.
This size value is based on the Reference:
Rana, R. K., Singhal, R., \& Dua, P. (2016). Deciphering the dilemma of parametric and nonparametric tests. Journal of the Practice of Cardiovascular Sciences, 2(2), 95.

The sample size to carried out any parametric and non-parametric analysis is 5, and it was established using common sense.
The warning and fails indicated in this section should be taking into account when a paper or report will be elaborated.



\end{document}
