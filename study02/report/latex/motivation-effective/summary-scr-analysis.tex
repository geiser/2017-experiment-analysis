
\documentclass[6pt,a4paper]{article}
\usepackage[a4paper,margin=0.54cm]{geometry}
\usepackage{longtable}
\usepackage{rotating}
\usepackage{pdflscape}
\usepackage{ctable}
\title{Statistical Analysis in the second study for students with effective participation }
\date{}
\begin{document}

\maketitle


%latex.default(result_df, caption = paste("Two-way ANOVA and Scheirer-Ray-Hare",     in_title), size = "small", longtable = T, ctable = F, landscape = F,     rowlabel = "", where = "!htbp", file = filename, append = T)%
\setlongtables{\small
\begin{longtable}{lrrrrrrrrrr}\caption{Two-way ANOVA and Scheirer-Ray-Hare in the second study for students with effective participation} \tabularnewline
\hline\hline
\multicolumn{1}{l}{}&\multicolumn{1}{c}{Sum Sq}&\multicolumn{1}{c}{Df}&\multicolumn{1}{c}{F value}&\multicolumn{1}{c}{Pr(\textgreater F)}&\multicolumn{1}{c}{Sig}&\multicolumn{1}{c}{Df}&\multicolumn{1}{c}{Sum Sq}&\multicolumn{1}{c}{H}&\multicolumn{1}{c}{p.value}&\multicolumn{1}{c}{Sig}\tabularnewline
\hline
\endfirsthead\caption[]{\em (continued)} \tabularnewline
\hline
\multicolumn{1}{l}{}&\multicolumn{1}{c}{Sum Sq}&\multicolumn{1}{c}{Df}&\multicolumn{1}{c}{F value}&\multicolumn{1}{c}{Pr(\textgreater F)}&\multicolumn{1}{c}{Sig}&\multicolumn{1}{c}{Df}&\multicolumn{1}{c}{Sum Sq}&\multicolumn{1}{c}{H}&\multicolumn{1}{c}{p.value}&\multicolumn{1}{c}{Sig}\tabularnewline
\hline
\endhead
\hline
\endfoot
\label{result}
Attention.(Intercept)&$ 797.236$&$ 1$&$512.642$&$0.000$&$$&$$&$$&$$&$$&$$\tabularnewline
Attention.Type&$   2.112$&$ 1$&$  1.358$&$0.250$&$$&$ 1$&$  425.506$&$1.933$&$0.164$&$$\tabularnewline
Attention.CLRole&$   0.013$&$ 1$&$  0.008$&$0.928$&$$&$ 1$&$    4.153$&$0.019$&$0.891$&$$\tabularnewline
Attention.Type:CLRole&$   1.626$&$ 1$&$  1.045$&$0.312$&$$&$ 1$&$  255.507$&$1.161$&$0.281$&$$\tabularnewline
Attention.Residuals&$  73.092$&$47$&$$&$$&$$&$47$&$10321.834$&$$&$$&$$\tabularnewline
Relevance.(Intercept)&$1157.488$&$ 1$&$793.890$&$0.000$&$$&$$&$$&$$&$$&$$\tabularnewline
Relevance.Type&$   0.417$&$ 1$&$  0.286$&$0.595$&$$&$ 1$&$  104.914$&$0.477$&$0.490$&$$\tabularnewline
Relevance.CLRole&$   0.276$&$ 1$&$  0.189$&$0.666$&$$&$ 1$&$   22.802$&$0.104$&$0.747$&$$\tabularnewline
Relevance.Type:CLRole&$   0.190$&$ 1$&$  0.130$&$0.720$&$$&$ 1$&$   28.383$&$0.129$&$0.719$&$$\tabularnewline
Relevance.Residuals&$  68.526$&$47$&$$&$$&$$&$47$&$10831.401$&$$&$$&$$\tabularnewline
Satisfaction.(Intercept)&$ 903.152$&$ 1$&$631.374$&$0.000$&$$&$$&$$&$$&$$&$$\tabularnewline
Satisfaction.Type&$   0.027$&$ 1$&$  0.019$&$0.891$&$$&$ 1$&$   23.395$&$0.107$&$0.744$&$$\tabularnewline
Satisfaction.CLRole&$   0.147$&$ 1$&$  0.103$&$0.750$&$$&$ 1$&$   36.149$&$0.165$&$0.685$&$$\tabularnewline
Satisfaction.Type:CLRole&$   0.941$&$ 1$&$  0.658$&$0.421$&$$&$ 1$&$  205.662$&$0.939$&$0.332$&$$\tabularnewline
Satisfaction.Residuals&$  65.801$&$46$&$$&$$&$$&$47$&$10682.794$&$$&$$&$$\tabularnewline
Level of Motivation.(Intercept)&$ 940.880$&$ 1$&$735.491$&$0.000$&$$&$$&$$&$$&$$&$$\tabularnewline
Level of Motivation.Type&$   0.925$&$ 1$&$  0.723$&$0.399$&$$&$ 1$&$  206.449$&$0.935$&$0.333$&$$\tabularnewline
Level of Motivation.CLRole&$   0.051$&$ 1$&$  0.040$&$0.843$&$$&$ 1$&$    0.265$&$0.001$&$0.972$&$$\tabularnewline
Level of Motivation.Type:CLRole&$   1.138$&$ 1$&$  0.889$&$0.350$&$$&$ 1$&$  206.977$&$0.938$&$0.333$&$$\tabularnewline
Level of Motivation.Residuals&$  60.125$&$47$&$$&$$&$$&$47$&$10620.809$&$$&$$&$$\tabularnewline
\hline
\end{longtable}}
\begin{flushright}{\scriptsize{Signif. codes: 0 ``**'' 0.01 ``*'' 0.05}}\end{flushright}


\begin{landscape}

%latex.default(result_df, caption = paste("Summary of Pair wilcoxon",     in_title), size = "small", longtable = T, ctable = F, landscape = F,     rowlabel = "", where = "!htbp", file = filename, append = T)%
\setlongtables{\small
\begin{longtable}{r}\caption{Summary of Pair wilcoxon in the second study for students with effective participation} \tabularnewline
\hline\hline
\multicolumn{1}{c}{}\tabularnewline
\hline
\endfirsthead\caption[]{\em (continued)} \tabularnewline
\hline
\multicolumn{1}{c}{}\tabularnewline
\hline
\endhead
\hline
\endfoot
\label{result}
$$\tabularnewline
\hline
\end{longtable}}

\end{landscape}

\section{Assumptions for Parametric Tests}
%latex.default(round_df(do.call(rbind, lapply(all_parametric_results,     function(p_results) {        return(do.call(rbind, lapply(p_results, function(p_result) {            return(data.frame(normality.fail = p_result$normality.fail,                 W = p_result$shapiro$statistic, p.value = p_result$shapiro$p.value))        })))    })), 3), caption = paste("Univariate normality test", in_title),     size = "scriptsize", longtable = T, ctable = F, landscape = F,     rowlabel = "", where = "!htbp", file = filename, append = T)%
\setlongtables{\scriptsize
\begin{longtable}{llrr}\caption{Univariate normality test in the second study for students with effective participation} \tabularnewline
\hline\hline
\multicolumn{1}{l}{}&\multicolumn{1}{c}{normality.fail}&\multicolumn{1}{c}{W}&\multicolumn{1}{c}{p.value}\tabularnewline
\hline
\endfirsthead\caption[]{\em (continued)} \tabularnewline
\hline
\multicolumn{1}{l}{}&\multicolumn{1}{c}{normality.fail}&\multicolumn{1}{c}{W}&\multicolumn{1}{c}{p.value}\tabularnewline
\hline
\endhead
\hline
\endfoot
\label{round}
Attention&FALSE&$0.979$&$0.501$\tabularnewline
Relevance&FALSE&$0.972$&$0.255$\tabularnewline
Satisfaction&FALSE&$0.957$&$0.067$\tabularnewline
Level of Motivation&FALSE&$0.982$&$0.629$\tabularnewline
\hline
\end{longtable}}

%latex.default(test_min_size_summary, caption = paste("Notes to be taken into account about sample size",     in_title), size = "scriptsize", longtable = T, ctable = F,     landscape = F, rowlabel = "", where = "!htbp", file = filename,     append = T)%
\setlongtables{\scriptsize
\begin{longtable}{lll}\caption{Notes to be taken into account about sample size in the second study for students with effective participation} \tabularnewline
\hline\hline
\multicolumn{1}{l}{}&\multicolumn{1}{c}{code}&\multicolumn{1}{c}{description}\tabularnewline
\hline
\endfirsthead\caption[]{\em (continued)} \tabularnewline
\hline
\multicolumn{1}{l}{}&\multicolumn{1}{c}{code}&\multicolumn{1}{c}{description}\tabularnewline
\hline
\endhead
\hline
\endfoot
\label{test}
Attention.Type.1&WARN: sample.size&current size is 13 and recommended size is 15 for the group: 'ont-gamified:Apprentice'.\tabularnewline
Attention.Type.2&WARN: sample.size&current size is 11 and recommended size is 15 for the group: 'non-gamified:Master'.\tabularnewline
Attention.Type.3&WARN: sample.size&current size is 8 and recommended size is 15 for the group: 'ont-gamified:Master'.\tabularnewline
Relevance.Type.1&WARN: sample.size&current size is 13 and recommended size is 15 for the group: 'ont-gamified:Apprentice'.\tabularnewline
Relevance.Type.2&WARN: sample.size&current size is 11 and recommended size is 15 for the group: 'non-gamified:Master'.\tabularnewline
Relevance.Type.3&WARN: sample.size&current size is 8 and recommended size is 15 for the group: 'ont-gamified:Master'.\tabularnewline
Satisfaction.Type.1&WARN: sample.size&current size is 13 and recommended size is 15 for the group: 'ont-gamified:Apprentice'.\tabularnewline
Satisfaction.Type.2&WARN: sample.size&current size is 11 and recommended size is 15 for the group: 'non-gamified:Master'.\tabularnewline
Satisfaction.Type.3&WARN: sample.size&current size is 8 and recommended size is 15 for the group: 'ont-gamified:Master'.\tabularnewline
Level of Motivation.Type.1&WARN: sample.size&current size is 13 and recommended size is 15 for the group: 'ont-gamified:Apprentice'.\tabularnewline
Level of Motivation.Type.2&WARN: sample.size&current size is 11 and recommended size is 15 for the group: 'non-gamified:Master'.\tabularnewline
Level of Motivation.Type.3&WARN: sample.size&current size is 8 and recommended size is 15 for the group: 'ont-gamified:Master'.\tabularnewline
\hline
\end{longtable}}

Recent studies carried out through simulations have indicated that ANOVA is reliable even when the data are non-normally distributed and the sample size is greater thatn 15 observations for each group.
This size value is based on the Reference:
Rana, R. K., Singhal, R., \& Dua, P. (2016). Deciphering the dilemma of parametric and nonparametric tests. Journal of the Practice of Cardiovascular Sciences, 2(2), 95.

The sample size to carried out any parametric and non-parametric analysis is 5, and it was established using common sense.
The warning and fails indicated in this section should be taking into account when a paper or report will be elaborated.



\end{document}
